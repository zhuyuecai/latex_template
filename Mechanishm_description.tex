\documentclass[12pt]{article}
\usepackage[T1]{fontenc}
\usepackage[utf8]{inputenc}
\usepackage{mathptmx}
%opening
\title{Template of Mechanism Description}
\author{Your Name}

\begin{document}

\maketitle

\section{Introduction} 
%Define the mechanism with a technical definition and add extensions necessary for the reader to understand the discussion.
%Technical definition: use one sentence to give a def to a term in the following manner:
%in what Context, the Term is Category for what Feature(usage or other stuff to make it different from othe term in the same Category) further extension(optional)
%Common Extensions:
%Comparison and contrast – when you need to show differences or similarities.  
%Classification – when you need to organize information into categories.
%Cause and Effect – when you need to demonstrate why something happens or when you need to trace results.
%Process – when you need to list the steps of a procedure
%Exemplification – when you need to give real or analogous examples.
%Etymology – to show the linguistic genesis of a term.   (limited use)


%Describe the mechanism’s overall function or purpose.
%Describe the mechanism’s overall appearance (shape, size, color, material, finish, texture, etc.)
%List the mechanism’s parts in the order in which they will be described.

   

\section{Discussion}
%Step 1 
%Define the first component part with a technical definition.
%Describe the part’s overall function or purpose.
%Describe the part’s shape, material, etc.
%Transition to the next part.
%Steps 2 to n
%Repeat as for step 1 for the remainder of the parts. 

\section{Conclusion}
%Summarize and provide closure.

%\bibliographystyle{ieeetr}
\bibliographystyle{acm}
\bibliography{paper}


\end{document}
